%%%%%%%%%%%%%%%%%%%%%%%%%%%%%%%%%%%%%%%%%%%%%%%%%%%%%%%%%%%%%%%%%%%%%%%
%% Abstract
\begin{abstract}
\thispagestyle{empty}

  Diese Arbeit befasst sich mit der Entwicklung und Implementierung eines autonomen Roboter Systems. Dieses System bildet das Szenario einer Objektübergabe zwischen zwei selbstständigen Manipulatoren ab. Bei diesen beiden Manipulatoren handelt es sich um den Typ-gleichen Roboterarm des YouBot der Firma Kuka. Einer der Arme ist dabei stationär auf einem Tisch fixiert, der andere auf der mobilen YouBot Plattform.
  
  Nach einer kurzen Einführung über das Nutzen von Robotern im Haushalt, sowie im intelligenten Gebäude, werden zunächst die Grundlagen und folgend ein aktueller Stand der Technik zusammengefasst. Anschließend wird sich diese Arbeit mit den zentralen Aspekten der Entwicklung und Implementierung befassen. Bei der Entwicklung wurden verschiedene Thematiken und Probleme der Robotik und Computer Vision berücksichtigt und bearbeitet. So werden die folgenden Kapiteln auf das Setup der Roboter eingehen, sowie den Themen der inversen Kinematik, der Segmentierung und Objekterkennung. Da das System in einem intelligenten Gebäude zum Einsatz kommen soll, ist die Thematik der Konfigurierung einzelner Subsysteme und Koordination ein Bestandteil dieser Arbeit. Die Implementierung befasst sich mit der Algorithmik der einzelnen Probleme. Den Abschluss der Arbeit bilden eine Beurteilung und ein Fazit der umgesetzten Lösung, auch wird ein Ausblick über mögliche zukünftige Weiterentwicklungen gegeben.
  
\end{abstract}




\begin{otherlanguage}{english}
	\begin{abstract}
		This paper considers the development and implementation of an autonomous robotic system. This system describes a scenario from a handover between two independent manipulators. These two manipulators are equal robot arms from the Kuka YouBot. One is stationary fixed at top of a table, the other one is on a mobile YouBot base.
		
		After a short introduction about the harness of service robots at smart houses, this work outlines the principles and the state of the art. Following this paper attends to the key aspects of the development and the implementation. The development observes different topics and a set of problems for robotic and computer vision. There will be a variety of subjects like roboter setup, inverse kinematic, segmentation and object recognition in the following chapters. Based on the operation at smart houses,  are konfiguration and koordinaten topics in this paper, too. The implementation sink in algorithmic and solutions for single problems. A valutation and a conclusion of the implemented solutions, and also a prospect of possible future projects, form the ending of this work. 
	\end{abstract}

\end{otherlanguage}


