%%%%%%%%%%%%%%%%%%%%%%%%%%%%%%%%%%%%%%%%%%%%%%%%%%%%%%%%%%%%%%%%%%%%%%%
%% Einleitung
\section{Einleitung}
\label{sec:einleitung}

Der Demographische Wandel in Deutschland stellt die Gesellschaft und die Politik vor ein großes Problem. Nicht nur der fehlende Nachwuchs, der ein Arbeitnehmerloch hinterlässt, sondern auch eine immer älter werdende Gesellschaft sorgen für viele Fragezeichen. Neben kommunalen Problemen, wie teure Infrastruktur, ist eins der größten Anliegen die Altersarmut. Ein sinkendes Rentenniveau und steigende Kosten werden es in Zukunft unmöglich machen für eine gute Pflegeversorgung zu bezahlen. \citep{brunozandonella2013} Neben den hohen Kosten in der Pflege sind auch andere Faktoren die zu einer nicht akzeptablen Situation führen. So ist der Pflegeberuf zu unattraktiv für viele junge Menschen, wodurch auch hier ein großes Loch an Arbeitnehmern\FemaleMale wahrzunehmen ist. Körperlich anstrengende Arbeit, die in kurzer Zeit ausgeführt werden muss, führen zu vielen physischen und psychischen Erkrankungen der Pfleger\FemaleMale .\citep{AOK2004}

Einen alternativen Weg bringt die Technik. Einfache zu bedienende Systeme können den Alltag vereinfachen und so auch älteren Menschen ein selbstständiges Leben ermöglichen. Alltägliche Aufgaben müssten nicht mehr von Pflegern übernommen werden, sondern könnten durch Maschinen erledigt werden. Schon heute können einzelne Kleinsysteme Haustätigkeiten wie Staubsaugen und Rasenmähen übernehmen. Ein Vorreiter auf diesem Gebiet ist das Roboterland Japan




