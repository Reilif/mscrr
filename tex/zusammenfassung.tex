%%%%%%%%%%%%%%%%%%%%%%%%%%%%%%%%%%%%%%%%%%%%%%%%%%%%%%%%%%%%%%%%%%%%%%%
%% Zusammenfassung und Ausblick
\section{Zusammenfassung und Ausblick}
Dieses Kapitel bildet den Abschluss dieser Arbeit.  Neben einer Zusammenfassung wird ein Ausblick über mögliche Weiterentwicklungen und Verbesserungen gegeben. Dabei wird auf die analysierten Schwachstellen des Systems eingegangen.

\subsection{Zusammenfassung}
In dieser Arbeit wurde ein Multi-Robot-System entwickelt. Dieses nutzt weitere Aktoren und Sensoren, ähnlich einem intelligenten Gebäude, zur Interaktion mit der Umwelt. Zentrale Komponenten dieses Systems sind zwei Roboter. Beide Robotersysteme bestehen aus einem Roboterarm. Der eine Roboter besitzt zudem eine mobilen Plattform, eine 3D-Kamera und einen Lasersensor. Als Anwendungsszenario wurde die Aufnahme durch den mobilen Roboter und die Übergabe an den stationären Roboter implementiert. Dabei wurden 3D-Kameras zur Objekterkennung und -Lokalisierung genutzt. Die Implementierung dafür nutzt Algorithmen aus der Computer Vision, genauer aus der Punktwolken Verarbeitung. Die dabei erreichte Genauigkeit der Sensoren und Algorithmen beträgt fünf Zentimeter bei der Raumüberwachung und fünf Millimeter bei der Nahfelderkennung. 

Zur effizienten Steuerung der Arme wurde iterativ eine eigene inverse Kinematik entwickelt. Diese beruht auf geometrischen Grundsätzen. Bei Tests zeigten diese eine Genauigkeit von circa vier Millimetern von den linearen und 0,6 Grad für den rotierenden Anteil der Zielpose. Für die Bewegung auf einer Trajektorie wurde ein Ansatz gewählt, der mit Hilfe von Wegpunkten den Ausschlag des Greifers verhindert. Nicht genau untersucht wurde die Entwicklung einer genauen Navigation der mobilen Plattform. Aus Gründen der Priorität wurde diese nur sehr rudimentär entwickelt. Bedingt durch ungenaue Odometriedaten bei Bewegungen entlang der Y-Achse kommt es zu hohen Ungenauigkeiten. Auch der Einsatz von Lasersensoren kann diesem nur bedingt entgegenwirken, da diese abhängig von einem Referenzpunkt der Odometrie sind. Andere Ansätze, wie SLAM, sind dabei viel genauer.

 Bei der Entwicklung des MRS wurde auf bestehenden Arbeiten, wie der PEIS-Ökologie, aufgebaut. Neben den Robotern kamen auch weitere eigenständige Sensoren und Aktoren zum Einsatz. RATS erweitert PEIS um die Möglichkeit für parallele und komplexere Koordinierung. Diese kann auch unter den einzelnen Agenten innerhalb eines RATS stattfinden. Da in dieser Arbeit die Anzahl an Robotern, Aktoren und Sensoren sehr gering ist, wurde die Konfiguration des MRS statisch implementiert. Eine Schnittstelle für eine dynamische Konfiguration ist im Konzept für RATS jedoch vorgesehen.

In dieser Arbeit konnte aus Platzgründen nicht auf weitere erarbeitete Konzepte eingegangen werden. Die Grundlagen für diese beruhen auf dieser Arbeit. Auch die Implementierung wurde im Rahmen dieser Arbeit durchgeführt. Dazu gehören Problematiken um den Greifer, der in dieser schriftlichen Ausarbeitung nur am Rande betrachtet wurde. Auf Grund mangelnder Griffigkeit des Greifer zum starren Testobjekt, kam es zum Einsatz des pinken Radiergummis. Dieses weißt einen hohen Reibungskoeffizienten und eine weiche, sowie flexible Struktur auf. Außerdem war es anhand der Farbe leicht zu identifizieren. Ein weiterer Randaspekt ist die Einbindung eines weiteren Aktors in das RATS-System. Dazu wurde im Rahmen dieser Arbeit eine farbige Lichtanlage installiert und implementiert, die dem Anwender oder anderen Personen im Labor ein visuelles Feedback über den Zustand des Robotersystems gibt.

\subsection{Ausblick}
\label{sec:ausblick}
Da der Umfang dieser Arbeit begrenzt ist, konnten nicht alle Ideen umgesetzt werden und wurden einigen Stellen angemerkt. Diese Ideen befassen sich mit der Erweiterbarkeit, aber auch mit der Verbesserung des Systems. Grundlegende Ansätze sind dabei der Greifer, die Koordinierung und Konfigurierung, die Autonomie des MRS und vor allem die Navigation, beziehungsweise Lokalisierung.

Der Greifer vom YouBot besteht aus zwei parallelen Fingern. Durch deren harte und glatte Beschaffenheit ist der Greifen von vielen Gegenständen nicht möglich. Auch die Erweiterung mit Softgrippern ließ keinen vernünftigen Griff zu. Eine erste Erweiterungsmöglichkeit stellt ein Greifer mit drei Fingern in Bogenform dar. Diese ermöglichen auch das Greifen von runden und kleinen Gegenständen. Die zweite Alternative besteht aus einem Greifer der aus einem Ballon, gefüllt mit feinkörnigem Granulat, besteht. Dieses umschließt das Zielobjekt und bildet bei Unterdruck eine feste Struktur, die das Objekt festhält. TODO Quelle

Für die Koordinierung und Konfigurierung existieren viele Arbeiten. Für das RATS empfiehlt sich eine Auktionsstruktur für das dynamische Zuweisen der Aufgaben, wie in TODO. Auch die Konfigurierung sollte einem dynamischen Ansatz folgen, damit zentrale Elemente wie der RATSCore entfallen können. Außerdem ist eine Umstellung von einem Router auf ein gemaschtes Netzwerk sinnvoll. Damit neue Agenten beitreten oder alte sich entfernen können. Eine Selbstbewertung des Systems und die automatische Anpassung von Konfigurationsparametern ist eine Möglichkeit die Autonomie des Systems zu erhöhen.

Alternativen für die Navigation und Lokalisierung gibt es viele. So ist ein Umstieg auf eine funkbasierte Lokalisierung, vergleichbar wie GPS, möglich. Auch Bild-basierte Ansätze, zum Beispiel durch die Raumüberwachung, sind denkbar. Am sinnvollsten ist vermutlich jedoch das SLAM-Verfahren, da vorhandene Lasersensoren genutzt werden können. Dieses würde die Genauigkeit des Systems stark steigern und damit auch eine Überarbeitung einzelner Konzepte ermöglichen.


