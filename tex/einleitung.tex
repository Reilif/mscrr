%%%%%%%%%%%%%%%%%%%%%%%%%%%%%%%%%%%%%%%%%%%%%%%%%%%%%%%%%%%%%%%%%%%%%%%
%% Einleitung
\section{Einleitung}
\label{sec:einleitung}

Der Demographische Wandel in Deutschland stellt die Gesellschaft und die Politik vor ein großes Problem. Nicht nur der fehlende Nachwuchs, der ein Arbeitnehmerloch hinterlässt, sondern auch eine immer älter werdende Gesellschaft sorgen für viele Fragezeichen. Neben kommunalen Problemen, wie teure Infrastruktur, ist eins der größten Anliegen die Altersarmut. Ein sinkendes Rentenniveau und steigende Kosten werden es in Zukunft unmöglich machen für eine gute Pflegeversorgung zu bezahlen. \citep{brunozandonella2013} Neben den hohen Kosten in der Pflege sind auch andere Faktoren die zu einer nicht akzeptablen Situation führen. So ist der Pflegeberuf zu unattraktiv für viele junge Menschen, wodurch auch hier ein großes Loch an Arbeitnehmern\FemaleMale wahrzunehmen ist. Körperlich anstrengende Arbeit, die in kurzer Zeit ausgeführt werden muss, führen zu vielen physischen und psychischen Erkrankungen der Pfleger\FemaleMale .\citep{AOK2004}

Einen alternativen Weg bringt die Technik. Einfache zu bedienende Systeme können den Alltag vereinfachen und so auch älteren Menschen ein selbstständiges Leben ermöglichen. Alltägliche Aufgaben müssten nicht mehr von Pflegern\FemaleMale übernommen werden, sondern könnten durch Maschinen erledigt werden. Schon heute können einzelne Kleinsysteme Haustätigkeiten wie Staubsaugen und Rasenmähen übernehmen. Ein Vorreiter auf diesem Gebiet ist das Roboterland Japan. Dort überlegte Kobayashi Hisato, Professor für Maschinenbau und Robotik an der Hosei-Universität in Tokio, sich schon 1999 Konzepte für Senioren-Service Roboter.\citep{wagner2009tele}  Nach seinen Vorstellungen sollten dabei Roboter nicht autonom arbeiten, sondern von Familienangehörigen ferngesteuert und überwacht.\citep{kobayashihisato1999} Neben diesen Konzepten kann man aber auch schon angewandte Robotik in Japans Seniorenpolitik finden. Zwei Musterbeispiele zeigen dabei die unterschiedlichen Anwendungsszenarien für Roboter im privaten Umfeld. \textit{Sinére K\={o}rien} der Firma Matsushita ist ein digitales Seniorenheim. Neben einer Smarthouse Anbindung gehört auch der Roboterteddy K\={o}-chan zur Ausstattung der einzelnen Zimmer. Dieser dient als Unterhaltungsroboter und Kommunikationsgerät mit den Pflegern\FemaleMale. So kann mit den Kameraaugen im Teddy eine Aufnahme vom Raum gemacht werden und im Notfall den Pflegern\FemaleMale eine Alarmmeldung geschickt werden. Weitere Sensoren, zu  Beispiel Gewichtssensoren unter den Betten, geben Informationen über die Abwesenheiten von Patienten\FemaleMale.\citep{wagner2009tele} Ein weiterer Anwendungsbereich für Roboter ist die \textit{robotto serap\={\i}} (Robotertherapie). Dabei beschäftigen sich die Senioren\FemaleMale mit tierähnlichen Robotern, wie Hunden, Seeroben oder Katzen. Im Zentrum der Therapie steht die Interaktion zwischen Patient\FemaleMale und Roboter. Die Robotertherapie soll die Patienten\FemaleMale aktivieren und deren Tagesabläufe abwechslungsreich gestalten. Außerdem steigert es die Kommunikation zwischen zwei Patienten\FemaleMale, die am selben Roboter arbeiten. \citep{wagner2009tele}

 \begin{figure}
 	\centering
 	\subfigure[Roboterteddy K\={o}-chan Quelle: \citep{panasonic2005}]{%
 		\includegraphics[scale=0.6]{fig/roboteddy}
 		\label{fig:roboTeddy}}
 	\hfill
 	\subfigure[Roboterkatze zur Therapie Quelle: \citep{wagner2009tele}]{%
 		\includegraphics[scale=0.4]{fig/robocat}
 		\label{fig:roboTeddy2}}
 	\caption{Roboter in Seniorenheimen}
 	\label{fig:robSenioren}
 \end{figure}

Nicht nur in Japan, sondern auch in Deutschland wird sich mit dem Thema \textit{Care(rob)bots} befasst. So finden sich unter dem Stichpunkt \textit{Mensch-Maschine-Entgrenzungen} Studien zu der Thematik. Andere Untersuchungen befassen sich mit der Gegenseite, der Akzeptanz der Senioren\FemaleMale für Roboter. So ergab eine Befragung der VDE-Studie "Mein Freund der Roboter", dass eine Mehrheit (56\%) der Senioren\FemaleMale Robotern im Haushalt offen gegenüber stehen und diese einem Pflege-/Altersheim vorziehen würden. Neben den bekannten Staubsauger- und Rasenmährobotern, sind es auch zukünftige Anwendung, wie ein roboterisierter Rollstuhl, die hohe Akzeptanzwerte erreichen. Die Studie zeigte aber auch, dass Senioren\FemaleMale zunächst Robotern skeptisch gegenüberstehen. Spontan lehnten 40 Prozent der Senioren\FemaleMale Roboter in ihrem privaten Umfeld ab, 60 Prozent empfanden Robotik sogar als unheimlich. jedoch zeigte sich, dass der Wunsch nach einer selbstständigen Lebensführung ein starker Faktor für die Akzeptanz ist. Dadurch ergibt sich eine Beliebtheit für Serviceroboter. So sind Roboter, die abgrenzbare Tätigkeiten im Haushalt selbständig erledigen, sehr beliebt. Wichtige Kriterien für die Akzeptanz waren zudem die intuitive Bedienbarkeit, die Robustheit und die Flexibilität gegenüber unterschiedlicher Handicaps. Auch menschliche Faktoren wie Geduld, Verständnis, Höflichkeit und Achtung der Intimsphäre waren den Anwendern\FemaleMale wichtig.\citep{dr.sibyllemeyer2011}



