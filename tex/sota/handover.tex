%%%%%%%%%%%%%%%%%%%%%%%%%%%%%%%%%%%%%%%%%%%%%%%%%%%%%%%%%%%%%%%%%%%%%%%
%% Related Work
\subsection{Grasping und Handover - Arbeiten zu Roboterarmen}
\label{sec:relatedwork-handover}
Dieses Kapitel befasst sich mit Arbeiten zu dem Thema Roboterarme. Zunächst wird das Greifen eines Gegenstandes, \textit{Grasping}, aufgezeigt. Dabei werden Arbeiten zum allgemeinen Greifen, zur Erkennung geeigneter Greifpositionen an Gegenständen und Kamera-unterstütztem Grasping aufgeführt. Im Anschluss wird die Thematik der Übergabe, dem zentralen Thema dieser Arbeit, aufgegriffen und Literatur zu Roboter-Mensch Übergaben betrachtet.

\subsubsection{Grasping - Greifen mit einem Roboterarm}
Als Ausgangsarbeit für dieses Thema bietet sich die Arbeit \cite{bicchi2000robotic} an. In dieser werden die Grundlagen des Greifens erläutert. Dabei wird zunächst die menschliche Hand und ihre Verwendungszwecke betrachtet. Der Mensch nutzt seine Hand zur Erkundung, zum Festhalten und zur Manipulation von Objekten. Das Erkunden von Objekten ist der eigene große Forschungsbereich \textit{Haptik} und kann bisher, auf Grund von fehlender Sensorik, in der Robotik nicht vollständig eingesetzt werden. Zwischen dem Festhalten, Fixieren und der Manipulation wird in der Arbeit unterschieden, sowie zwischen dem Manipulieren mit den Fingern und der Manipulation mit dem gesamten Arm. Dies liegt vor allem an den Anwendungs- und Forschungsbereichen. Während das Fixieren eines Objektes, beziehungsweise die Manipulation mit dem gesamten Arm, in der Industrie oft eingesetzt wird, ist die filigrane Arbeit mit den Fingern noch ein Thema der Forschung. Die ersten Arbeiten zu diesem Thema der Robotik gehen auf \cite{asada1979studies}  und \cite{mason1985robot} zurück. Neben dem Greifen mit den Fingern der Hand gibt es auch Griffe mit dem kompletten Arm, bezeichnet mit \textit{whole arm graps} \citep{townsend1988effect}, \citep{bicchi1994problem} und \textit{power grasps} (\citep{mirza1990force}) \citep{bicchi2000robotic}. 

Das Greifen unterliegt physischen Grenzen. Dabei wird der Griff, beziehungsweise das Halten, unter anderem durch den Kraftvektor und dem Haftkoeffizienten beeinflusst. Ein Griff wird durch $N$ Kontakte beschrieben. Dabei wird angenommen, dass alle Kontakte punktuell sind. Auch ein Kontakt auf einer Linie oder Oberfläche wird durch zwei oder mehr Punktkontakte abgebildet. Die Literatur unterscheidet diese Kontakte in reibungslose, reibungsbedingte oder weiche Punktkontakte \citep{salisbury1983kinematic}. Ein reibungsloser Kontakt kann nur eine Kraft entlang der gemeinsamen Normalen erzeugen. Ein reibungsbedingter kann neben der normalen auch eine tangentiale Kraft und ein weicher Kontakt zusätzlich ein Drehmoment erzeugen. Die Kontaktart ist abhängig von den Oberflächeneigenschaften des Greifers und des Objektes. Bei einer gummiähnlichen Oberfläche des Greifers wird der Kontakt als weich modelliert. Haben Greifer und Objekt beide harte und raue Oberflächen wird ein reibungsbedingter Kontakt angenommen. Sind die Kontaktstellen auf Greifer und Objekt glatt und ist ein kleiner Reibungskoeffizient gegeben, gilt der Kontakt als reibungslos \citep{bicchi2000robotic}.


Neben dem Greifen selbst existiert auch die Problematik des Greifpunktes. Also der Stelle am Objekt, an welcher der Greifer ansetzt. Dazu existieren verschiedene Arbeiten. Die ersten zu diesem Thema waren die Arbeiten \cite{kamon1996learning}, \cite{coelho2001developing} und \cite{bowers2003manipulation}. In diesen werden mit Hilfe von Sensoren planare 2D-Objekte gegriffen. Ergebnisse mit 3D-Objekten erreichte die Arbeit \cite{saxena2008robotic}. Dabei werden zunächst mit maschinellem Lernen und gelabelten Testdatensätzen gute Griffpunkte für 3D-Objekte angelernt. Anschließend kann der Algorithmus unbekannte 3D-Objekte bewerten und die besten Griffpunkte finden. Diese Grundlagenforschung wird in der Arbeit \cite{maitin2010cloth} genutzt um an unbekannten 3D-Objekten aus Stoff Ecken zum Greifen und Zusammenlegen zu finden. In dieser werden vor allem Ansätze nach dem RANSAC-Verfahren gewählt, um bestimmte Strukturen gezielt zu finden. Für weitere Informationen lassen sich in der Literatur noch viele verschiedene Anwendungsfälle und Ansätze zum Greifen finden.

\subsubsection{Handover - Übergabe zwischen zwei Systemen}
Die Übergabe zwischen zwei Systemen ist eine häufige Interaktion im Alltag. Dieses betrifft oft die Manipulation eines Objektes. Die Arbeit \cite{huber2008human} beschäftigt sich mit der Übergabe zwischen Mensch und Roboter und beinhaltet zunächst eine Analyse einer Übergabe zwischen zwei Menschen. Dabei stellt sich folgende Aktionsreihenfolge für eine Übergabe heraus \citep{huber2008human}:

\begin{enumerate}
	\item Der Geber nimmt das Zielobjekt
	\item Der Geber bewegt die Hand Richtung Übergabeposition
	\item Der Nehmer bewegt die Hand Richtung Übergabeposition
	\item Transaktion
	\item Beide nehmen ihre Hände zurück
\end{enumerate}

Eigene Beobachtungen vom Mensch-Mensch Übergaben ergaben, dass diese Form der Übergabe nur eine Interaktionsart ist und als \textit{Geben} bezeichnet werden kann, da die Interaktion vom Geber ausgeht. Eine Alternative dazu stellt das \textit{Nehmen} dar, bei dem die Interaktion vom Nehmer ausgeht. Eine optimierte dritte Interaktionsart wäre eine synchrone Bewegung beider Akteure, bei der eine Verzögerung durch die Reaktionszeit des Interaktionspartners entfällt.

In der Robotik existieren mehrere Arten der Übergabe. Diese unterscheiden sich anhand des Interaktionspartners und des Anwendungsfeldes. 
In der Industrie, zum Beispiel dem Automobilbau, werden die Werkstücke nicht direkt zwischen den einzelnen Robotern übergeben, sondern befinden sich auf einem Förderband. Dieses fährt das Werksstuck auf eine definierte Position und die einzelnen Roboter führen ihren Arbeitsschritt aus. Anschließend wird das Werksstück zur nächsten Station gebracht. Eine weitere Art der Übergabe ist die Mensch-Roboter-Interaktion. Dieses Thema ist ein weitverbreitetes Thema mit vielen Aspekten. So beschäftigen sich die Arbeiten \cite{huber2008human} und \cite{shibata1995experimental} mit dem Timing-Verhalten, \cite{mainprice2010planning} und \cite{kulic2005safe} befassen sich mit der sicheren Planung von Übergabe-Interaktionen. Weitere Arbeiten (unter anderem \cite{prada2014implementation} und \cite{basiliapproach}) beschäftigen sich mit den Reaktionen auf menschliche Aktionen, wie Gesten (zum Beispiel: Handfläche nach oben geöffnet als Geste fürs Nehmen) und Veränderungen während der Übergabe.

In dieser Arbeit steht jedoch die Roboter-Roboter-Übergabe im Fokus. Dieses Thema ist in der Literatur nicht vorhanden oder wird nur als Randaspekt erwähnt.